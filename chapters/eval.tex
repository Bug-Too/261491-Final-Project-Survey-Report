\chapter{\ifproject%
\ifenglish Experimentation and Results\else การทดลองและผลลัพธ์\fi
\else%
\ifenglish System Evaluation\else การประเมินระบบ\fi
\fi}

ในบทนี้จะทดสอบเกี่ยวกับการทำงานในฟังก์ชันหลักๆ

\section{ทดสอบประสิทธิภาพของโปรแกรม}

ปัจจัยที่ใช้ในการทดสอบประสิทธิภาพของโปรแกรม ได้แก่
\begin{enumerate}
    \item Central Processing Unit (CPU)
    \item Graphics Processing Unit (GPU)
    \item Memory
    \item Input lag
    \item Network latency
    \item Frame per second (FPS)
\end{enumerate}

สามารถทดสอบได้โดยใช้ Profiler ของ Unreal Engine ซึ่งแสดงข้อมูลเกี่ยวกับ frame rates การใช้งานของ CPU Memory และ GPU ในรูปแบบของกราฟและ call stacks

\section{ทดสอบความพึงพอใจของกลุ่มเป้าหมาย}

กลุ่มเป้าหมายที่ คือ วัยรุ่นและวัยผู้ใหญ่ที่มีอายุมากกว่า 18 ปีขึ้นไป อย่างน้อย 5-10 คน การทดสอบทั้งหมดจะแบ่งออกเป็น 5 ส่วน เป็นข้อมูลเชิงคุณภาพ ได้แก่
\begin{enumerate}
    \item Gameplay mechanics ในส่วนนี้เราจะทดสอบ gameplay ในช่วงพัฒนา prototype ทดลองเล่นกันเอง และ ให้กลุ่มเป้าหมาย แสดงความคิดเห็นเกี่ยวกับตัวเกมแล้วปรับปรุงตัวเกม
    \item Difficulty ความยากของตัวเกมทดสอบโดยให้กลุ่มเป้าหมาย ลองเล่นเกมและ ตอบแบบสอบถามด้านความรู้สึกเมื่อ ตัวละครตาย ระยะเวลาที่ใช้ในการเล่นเกม และ การหาไอเทมในเกม
    \item Player's roles balance ความสมดุลของผู้เล่นทั้ง 2 ทดสอบโดยให้กลุ่มเป้าหมาย แบ่งเป็น 2 กลุ่มโดยเล่นในบทบาทที่แตกต่างกัน  ทดลองเล่นเกม และ ตอบแบบสอบถามด้านความรู้สึกเมื่อ ตัวละครตาย ระยะเวลาที่ใช้ในการเล่นเกม และ การหาไอเทมในเกม
    \item Emotion delivered อารมณ์ที่ตัวเกมสื่อให้กับผู้เล่น ทดสอบโดยให้กลุ่มเป้าหมายทดลองเล่น สอบถามและสังเกตอารมณ์ของผู้เล่นขณะเล่นเกม และ ให้ผู้เล่นตอบแบบสอบถาม ด้านอารมณ์ เครียด กลัว หวาดระแวง และ ขบขัน หลังจากเล่นเกมเสร็จแล้ว
    \item Replay value ของตัวเกมทดสอบโดยให้กลุ่มเป้าหมาย ลองเล่นเกมและ ตอบแบบสอบถามความสนุก และ ความต้องการที่จะเล่นเกมซ้ำ
\end{enumerate}