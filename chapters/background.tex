\chapter{\ifenglish Background Knowledge and Theory\else ทฤษฎีที่เกี่ยวข้อง\fi}

การทำโครงงาน เริ่มต้นด้วยการศึกษาค้นคว้า ทฤษฎีที่เกี่ยวข้อง หรือ งานวิจัย/โครงงาน ที่เคยมีผู้นำเสนอไว้แล้ว ซึ่งเนื้อหาในบทนี้ก็จะเกี่ยวกับการอธิบายถึงสิ่งที่เกี่ยวข้องกับโครงงาน เพื่อให้ผู้อ่านเข้าใจเนื้อหาในบทถัดๆ ไปได้ง่ายขึ้น

\section{เกมสยองขวัญแบบผู้เล่นหลายคนในตลาด}
\subsection{Dead Space 3}

Dead Space 3 เป็นเกมสยองขวัญที่ให้ผู้เล่นสามารถเล่นโดยใช้ความร่วมมือจากผู้เล่น 2 คน โดยต่อสู้กับ Necromorph โดยรวมแล้วเป็นเกมสยองขวัญเอาชีวิตรอดที่ตึงเครียดและน่าตื่นเต้นที่นำเสนอประสบการณ์ความร่วมมือที่สมจริง เรื่องราวที่น่าดึงดูด และกลไกการเล่นเกมใหม่ที่หลากหลายเพื่อให้ผู้เล่นมีส่วนร่วมและลุ้นตาม

\subsection{Left 4 Dead 2}

Left 4 Dead 2 เป็นเกมสยองขวัญเอาชีวิตรอดจากโลกที่เต็มไปด้วยฝูงผีดิบ โดยตัวเกมสามารถให้ผู้เล่น ด้วยกันได้ตั้งแต่ 1 ถึง 4 คนโดยผู้เล่นจะรับบทบาทเป็นผู้เอาชีวิตรอดโดยตัวเกมจะมีอาวุธให้ผู้เล่นได้เลือกตามความถนัดของตน โดยรวมแล้วเป็นเกมยิงแบบร่วมมือที่รวดเร็วและเข้มข้น ซึ่งนำเสนอการผสมผสานระหว่างกลยุทธ์ การทำงานเป็นทีม และ ความสยองขวัญเอาชีวิตรอด ด้วยรูปแบบการเล่นที่น่าดึงดูด ตัวละครที่น่าจดจำ

\subsection{Phasmophobia}

Phasmophobia เกมนี้ให้ผู้เล่นสวมบทบาทเป็นนักล่าผี และ รองรับผู้เล่นสูงสุด 4 คนที่ทำงานร่วมกันเพื่อสำรวจสถานที่ และ รวบรวมหลักฐานเหตุการณ์เหนือธรรมชาติ โดยรวมแล้วเป็นเกมสยองขวัญที่ไม่เหมือนใครและน่าดึงดูด ซึ่งรวมเอาองค์ประกอบของการสืบสวน การเอาชีวิตรอด และการเล่นเกมแบบร่วมมือกัน เป็นเกมที่ต้องใช้การสื่อสาร การทำงานเป็นทีม และการคิดเชิงกลยุทธ์เพื่อให้ประสบความสำเร็จ 

\subsection{The Forest}

The Forest เกมสยองขวัญเอาชีวิตรอด open world โดยผู้เล่นทำงานร่วมกันเพื่อสร้างที่พักพิง รวบรวมทรัพยากร และ ป้องกันมนุษย์กลายพันธุ์ โดยรวมแล้วเป็นเกมสยองขวัญเอาชีวิตรอดที่น่าตื่นเต้นและท้าทายที่นำเสนอการผสมผสานที่ไม่เหมือนใครระหว่างการสำรวจ การสร้าง และการต่อสู้ ด้วยบรรยากาศที่ตึงเครียด ศัตรูที่ท้าทาย และโลกที่สมจริง 

\section{\ifenglish%
\ifcpe CPE \else ISNE \fi knowledge used, applied, or integrated in this project
\else%
ความรู้ตามหลักสูตรซึ่งถูกนำมาใช้หรือบูรณาการในโครงงาน
\fi
}

\begin{enumerate}
  \item Software Engineering: ใช้ความรู้เรื่องการจัดการการพัฒนาซอฟต์แวร์
  \item Computer Networking: ใช้ความรู้ในเรื่องเครือข่ายคอมพิวเตอร์ในการพัฒนาส่วนการเล่นหลายคน
  \item Object-oriented Programming: ใช้ความรู้ในการเขียนโปรแกรมเชิงวัตถุในการเขียนโปรแกรม
\end{enumerate}


\section{\ifenglish%
Extracurricular knowledge used, applied, or integrated in this project
\else%
ความรู้นอกหลักสูตรซึ่งถูกนำมาใช้หรือบูรณาการในโครงงาน
\fi
}

\begin{enumerate}
  \item การพัฒนาเกมด้วย Unreal Engine โดยใช้ Blueprint และภาษา C++
  \item การออกแบบวิธีการเล่นและฉาก
  \item การออกแบบงานศิลป์ในเกม
  \item การสร้างเนื้อเรื่องและเนื้อหาในเกม
\end{enumerate}

