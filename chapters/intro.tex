\chapter{\ifenglish Introduction\else บทนำ\fi}

\section{\ifenglish Project rationale\else ที่มาของโครงงาน\fi}
แนวเกมสยองขวัญเป็นแนวเกมที่ได้รับความนิยมในปัจจุบัน หลายเกมเป็นเกมแบบผู้เล่นคนเดียว ซึ่งสามารถมอบประสบการณ์ที่เร้าใจ น่ากลัว ให้กับผู้เล่น อีกทั้งยังทำให้ผู้เล่นมีประสบการณ์ร่วมที่ดี แต่ทว่าเกมแบบผู้เล่นคนเดียวขาดเรื่องการปฏิสัมพันธ์กับผู้เล่นคนอื่น ผู้เล่นไม่สามารถมีประสบการณ์ร่วมกับผู้อื่นได้ และเกมแบบผู้เล่นคนเดียวยังเน้นด้านการเล่าเรื่อง ซึ่งบ่อยครั้งผู้เล่นมักจะไม่กลับมาเล่นซ้ำใหม่ ทางผู้พัฒนาจึงสนใจที่จะพัฒนาเกมสยองขวัญที่อาศัยความร่วมมือของผู้เล่นสองคนในการฝ่าฟันอุปสรรคในรูปแบบที่เกมสยองขวัญแบบผู้เล่นคนเดียวทำไม่ได้ และดึงดูดให้ผู้เล่นสามารถสนุกกับการกลับมาเล่นซ้ำใหม่ได้

\section{\ifenglish Objectives\else วัตถุประสงค์ของโครงงาน\fi}
\begin{enumerate}
    \item เพื่อสร้างความบันเทิง โดยผู้เล่นสามารถกลับมาเล่นซ้ำได้โดยไม่รู้สึกเบื่อ
    \item เพื่อสร้างเสริมความร่วมมือและการสื่อสารที่ดีและสนุกของผู้เล่น
\end{enumerate}

\section{\ifenglish Project scope\else ขอบเขตของโครงงาน\fi}

\subsection{\ifenglish Hardware scope\else ขอบเขตด้านฮาร์ดแวร์\fi}
\begin{enumerate}
    \item โปรแกรมสามารถรองรับตัวควบคุมได้สามช่องทาง (คีย์บอร์ด เมาส์ และคอลโทรเลอร์)
\end{enumerate}

\subsection{\ifenglish Software scope\else ขอบเขตด้านซอฟต์แวร์\fi}
\begin{enumerate}
    \item โปรแกรมสามารถเล่นได้ผ่านทางคอมพิวเตอร์ระบบปฏิบัติการ Windows เท่านั้น
\end{enumerate}

\section{\ifenglish Expected outcomes\else ประโยชน์ที่ได้รับ\fi}

ผู้เล่นเกมสนุกสนานกับการเล่นเกมนี้ร่วมกับผู้อื่น ส่งเสริมให้ผู้เล่นมีปฏิสัมพันธ์ที่ดีในการเล่นเกมนี้ร่วมกัน

\section{\ifenglish Software technology\else เทคโนโลยีด้านซอฟต์แวร์\fi}
\begin{enumerate}
    \item Unreal Engine: เกมเอนจินที่ใช้พัฒนาวิดีโอเกม
    \item Perforce: ระบบจัดการ version ของการพัฒนาโปรแกรม
\end{enumerate}

\section{\ifenglish Project plan\else แผนการดำเนินงาน\fi}

\begin{plan}{12}{2022}{3}{2024}
    \planitem{12}{2022}{03}{2023}{รวบรวมความต้องการของผู้เล่น}
    \planitem{12}{2022}{03}{2023}{ออกแบบวิธีการเล่น}
    \planitem{01}{2023}{03}{2023}{สร้างต้นแบบ}
    \planitem{03}{2023}{05}{2023}{ขัดเกลาวิธีการเล่น}
    \planitem{03}{2023}{05}{2023}{พัฒนาและทดสอบต้นแบบ}
    \planitem{03}{2023}{05}{2023}{ออกแบบระบบของโปรแกรม}
    \planitem{06}{2023}{06}{2023}{สร้างสรรค์งานศิลป์}
    \planitem{06}{2023}{06}{2023}{เขียนโปรแกรมส่วนการเล่น}
    \planitem{11}{2023}{12}{2023}{สร้างสรรค์งานศิลป์ (ต่อ)}
    \planitem{11}{2023}{12}{2023}{เขียนโปรแกรมส่วนการเล่น (ต่อ)}
    \planitem{11}{2023}{12}{2023}{เขียนโปรแกรมส่วน UI}
    \planitem{11}{2023}{12}{2023}{นำงานศิลป์มาประกอบกัน}
    \planitem{12}{2023}{01}{2024}{งานเสียงและดนตรีประกอบ}
    \planitem{12}{2023}{01}{2024}{ทดสอบเกม}
    \planitem{02}{2024}{03}{2024}{วัดและรายงานผล}
\end{plan}

\section{\ifenglish Roles and responsibilities\else บทบาทและความรับผิดชอบ\fi}
\begin{enumerate}
    \item นายธนดล เดชประภากร หน้าที่: Gameplay design, Game programmer, Sound design, Testing
    \item นายภูริช สีนวลแล หน้าที่: Gameplay design, UI programmer, UI/UX designer, Game assets designer, Testing
\end{enumerate}

\section{\ifenglish%
Impacts of this project on society, health, safety, legal, and cultural issues
\else%
ผลกระทบด้านสังคม สุขภาพ ความปลอดภัย กฎหมาย และวัฒนธรรม
\fi}

โปรแกรมนี้สามารถสร้างเสริมทักษะความร่วมมือ การสื่อสาร และการแก้ไขปัญหาเฉพาะหน้าพร้อมความบันเทิงควบคู่กัน
