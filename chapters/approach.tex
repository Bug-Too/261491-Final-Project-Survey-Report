\chapter{\ifproject%
\ifenglish Project Structure and Methodology\else โครงสร้างและขั้นตอนการทำงาน\fi
\else%
\ifenglish Project Structure\else โครงสร้างของโครงงาน\fi
\fi
}

ในบทนี้จะกล่าวถึงหลักการ และการออกแบบระบบ โดยการออกแบบเกมได้ใช้ Game Design Template Master Doc \cite{GDD1, GDD2} เป็นต้นแบบแบบหลวม ๆ ซึ่งการพัฒนาจะเป็นแบบ iterative process และมีการสร้าง prototype เพื่อทดสอบวิธีการเล่นและกลไกว่าเป็นไปตามที่ต้องการหรือไม่

\makeatletter

% \renewcommand\section{\@startsection {section}{1}{\z@}%
%                                    {13.5ex \@plus -1ex \@minus -.2ex}%
%                                    {2.3ex \@plus.2ex}%
%                                    {\normalfont\large\bfseries}}

\makeatother
%\vspace{2ex}
% \titleformat{\section}{\normalfont\bfseries}{\thesection}{1em}{}
% \titlespacing*{\section}{0pt}{10ex}{0pt}

\section{แนวคิดหลักของเกม (Core)}

\subsection{Design Pillars}

เป็นรากฐานของเกมที่จะไม่เปลี่ยนแปลง และการออกแบบส่วนต่าง ๆ ของเกมควรสอดรับกับสิ่งเหล่านี้ \cite{GDD1}

\begin{enumerate}
  \item เกมสยองขวัญที่ผู้เล่นต้องช่วยกันถึงจะผ่านไปได้
  \item อีกคนนึงเห็นปีศาจ อีกคนนึงไม่เห็น
  \item เกมมีความท้าทายในทุกครั้งที่เล่น สามารถเล่นซ้ำได้โดยไม่รู้สึกเบื่อ
\end{enumerate}

\subsection{Game Loops}

ผู้เล่นทั้งสองต้องช่วยกันทำลายวงแหวนเวทย์มนต์ตามที่ต่าง ๆ พร้อมเอาชีวิตรอดจากปีศาจ

\subsection{แรงจูงใจในการเล่น (Player Motivation)}

ผู้เล่นมีแรงจูงใจในการเล่นเพื่อเอาชนะปีศาจ และเมื่อเล่นต่อไปหลังจากจบหนึ่งเกมหรือขณะเล่นอยู่ เกมจะยากขึ้นและบีบให้ผู้เล่นเล่นในแบบที่ยังไม่เคยเล่นมาก่อน

\section{เนื้อหา}

\subsection{โดยรวม}

เกมนี้เป็นเกมสยองขวัญที่อาศัยความร่วมมือของผู้เล่นสองคน เป็นเกมมุมมองบุคคลที่หนึ่ง งานภาพเป็นแนว 3D Photorealistic หนึ่งเกมสามารถเล่นจบภายในเวลา 30-40 นาที

\subsection{เนื้อเรื่อง}

เกมนี้มีเนื่อเรื่องและสถานที่อยู่ในยุโรปยุคกลาง ตามความเชื่อที่ว่า ปีศาจสามารถปลอมแปลงเป็นสัตว์ชนิดหนึ่ง อาศัยอยู่ในป่ารกร้าง หน้าที่ของผู้เล่นคือ ต้องไปปราบปีศาจร้ายและทำให้ผู้คนปลอดภัย

\subsection{ตัวละคร}

ผู้เล่นสามารถเลือกเล่นเป็นแม่ชี (Nun) หรือนายพราน (Hunter) ซึ่งแต่ละตัวละครมีความสามารถแตกต่างกันไป

\subsubsection{ตัวละครนายพราน (Hunter)}

\paragraph{ความสามารถ}
\begin{itemize}
  \item สามารถปกป้องแม่ชีจากกับดักในป่าได้
  \item สามารถรักษาตัวเองและแม่ชี โดยใช้สมุนไพรในป่า
  \item มีแผนที่ใช้นำทาง สามารถทำเครื่องหมายบนแผนที่ได้
\end{itemize}

\paragraph{การออกแบบ}
\begin{itemize}
  \item เป็นชาวยุโรปผิวขาว
  \item มีหนวดเคราและผมสีดำยาว
  \item อายุประมาณ 40 ปี
  \item สวมชุดนายพราน
\end{itemize}

\subsubsection{ตัวละครแม่ชี (Nun)}

\paragraph{ความสามารถ}
\begin{itemize}
  \item สามารถป้องกันการโจมตีจากปีศาจ
  \item สามารถทำลายวงแหวนเวทย์มนต์ได้
  \item มีสายตาที่แย่ไม่สามารถเห็นได้ในระยะไกล
  \item แต่สามารถเห็นปีศาจในระยะไกล และรู้ว่าปีศาจคือสัตว์ตัวไหน
  \item สามารถรับรู้ได้ถึงวงแหวนเวทย์มนต์จากระยะไกล
\end{itemize}

\paragraph{การออกแบบ}
\begin{itemize}
  \item เป็นชาวยุโรปผิวขาว
  \item สวมชุดแม่ชี
  \item อายุประมาณ 35 ปี
\end{itemize}

\subsubsection{ปีศาจ}

ปีศาจเป็นสัตว์หน้าตาดูปกติ อาศัยอยู่ในป่า แฝงตัวอยู่ตามฝูงสัตว์ปกติอื่น ๆ สัตว์ที่เป็นปีศาจได้ ประกอบไปด้วย
\begin{itemize}
  \item กระต่าย
  \item สุนัขจิ้งจอก
  \item หมาป่า
  \item หมู
\end{itemize}

การที่ปีศาจหน้าตาดูปกติและนายพรานไม่สามารถรู้ได้ว่าสัตว์ตัวไหนเป็นตัวร้ายท่ามกลางสัตว์ปกติในเกม ทำให้ต้องมีการสื่อสารกันระหว่างผู้เล่น ทำให้เกิดอารมณ์ระแวง เพิ่มความตึงเครียด และความท้าทายให้กับเกม

\subsection{ฉาก}

สถานที่ของเกม คือ ป่ารกร้างในยุโรปยุคกลาง ต้นไม้ในฉากจะไม่มีใบหรือใบแห้งแร้งมาก บนพื้นจะเต็มไปด้วยหญ้าแห้งและใบไม้แห้ง ให้อารมณ์ความรู้สึกที่โดดเดี่ยว ไร้ชีวิตชีวา ทั้งในยามกลางวันและกลางคืน ต้นไม้ประกอบฉากต้องเป็นต้นไม้ในป่ายุโรป เช่น European Hornbeam หรือ Black Alder ภูมิประเทศเป็นพื้นที่ราบมีเนินและแหล่งน้ำบ้าง เพื่อเพิ่มความน่าสนใจและมีจุดสังเกตุให้ผู้เล่น

ฉากของเกมควรมีขนาดที่ใหญ่เหมาะสมที่ให้ผู้เล่นสามารถสำรวจสถานที่ทั้งหมดได้ภายในเวลาไม่เกิน 20 นาที จากระยะเวลาการเล่นทั้งหมด 30-40 นาที

\subsection{เสียงและดนตรีประกอบ}

ดนตรีประกอบเป็นดนตรีตะวันตก ใช้เครื่องดนตรียุโรปยุคกลางผสมสมัยคลาสสิค จะต้องมีดนตรีในช่วงที่ต่อสู้กับปีศาจ ช่วงที่กำลังทำลายวงแหวนเวทย์มนต์ และช่วงสำคัญอื่น ๆ โดยรวมแล้วดนตรีประกอบต้องมีอารมณ์ที่ลุ้นระทึก แต่ไม่ทำลายบรรยากาศที่เหงา น่ากลัว ของเกม อาจมีเสียงดนตรีอื่นที่ใช้สร้างบรรยากาศ เช่น เสียงดนตรี Drone เสียงโน้ตเปียโนที่เล่นแบบย้อนกลับ เป็นต้น

เสียงประกอบที่จำเป็นต้องมี คือ เสียงเดินของตัวละคร เสียงวิ่งของตัวละคร เสียงหอบของผู้เล่น
เสียงเจ็บและตายของตัวละคร เสียงร้องปกติของสัตว์

\section{คุณสมบัติ (Features)}

\subsection{การสร้างฉาก}

การสร้างฉากในการเล่นแต่ละครั้ง จะเปลี่ยนตำแหน่งและจำนวนของสมุนไพร ตำแหน่งของกับดัก และตำแหน่งของวงแหวนเวทย์มนต์ รวมถึงในการเล่นแต่ละครั้ง ปีศาจจะอยู่ในร่างสัตว์ที่ไม่เหมือนกัน ทั้งนี้การจัดวางสิ่งของต้องมีความสมดุล มีจำนวนที่เหมาะสมที่สามารถเล่นผ่านไปได้อย่างสนุก ไม่ควรกองกันอยู่ที่เดียวกันเยอะเกินไปเพราะจะทำให้เกมจบเร็วขึ้น ไม่สนุก การเปลี่ยนตำแหน่งและจำนวนของสิ่งของจะเพิ่มความท้าทาย และความยากในการเล่นแต่ละรอบได้ในแบบที่ไม่ซ้ำกัน  

\subsection{กลางวันและกลางคืน}

ช่วงเวลากลางวันจะเป็นช่วงเย็น ตอนกลางวันผู้เล่นจะไม่สามารถรู้ว่าปีศาจคือสัตว์ตัวไหน แต่ปีศาจสามารถปรากฏตัวได้อยู่ วงแหวนเวทย์จะยังไม่ปรากฏ ช่วงกลางวันมีไว้เพื่อให้ผู้เล่นสำรวจสถานที่ที่น่าสนใจและเก็บรวบรวมยา สามารถใช้โอกาสนี้สังเกตุสัตว์ที่อยู่ในป่า ปลดกับดัก และทำสัญลักษณของจุดที่สนใจในแผนที่ (Farm and Explore)

ช่วงเวลากลางคืนเป็นช่วงที่ปีศาจสามารถทำร้ายเราได้ วงแหวนเวทย์มนต์จะปรากฏขึ้น ผู้เล่นต้องทำหน้าที่ช่วยกันทำลายวงแหวนเวทย์และเอาตัวรอดจากปีศาจ ช่วงนี้จะเป็นช่วงสุดท้ายของเกม (Climax)

ทั้งสองช่วงจะมีการเปลี่ยนผ่านที่เหมือนกับโลกจริง คือ ดวงอาทิตย์เคลื่อนที่ลาลับขอบฟ้า ทั้งสองช่วงกินเวลาเท่ากัน คือ 20 นาที

\subsection{การสิ้นสุดเกม}

ผู้เล่นสามารถเอาชนะปีศาจได้เมื่อทำลายวงแหวนเวทย์มนต์ได้ครบทุกอัน และจะแพ้เมื่อผู้เลนคนใดคนนึงเสียชีวิตหรือทั้งคู่ไม่สามารถทำลายวงแหวนเวทย์มนต์ในเวลาที่กำหนดได้

\subsection{การทำลายวงแหวนเวทย์มนต์}

ผู้เล่นสามารถทำลายวงแหวนเวทย์มนต์โดยใช้ความสามารถพิเศษของแม่ชี การทำลายวงแหวนใช้เวลาไม่เกิน 1 นาที และจะเพิ่มความดุร้ายของปีศาจในรูปแบบของการเพิ่มความถี่ในการปรากฎตัว เพื่อให้เกิดความตึงเครียด และเพิ่มความเข้มข้น

\subsection{การเดินและการเคลื่อนไหวของตัวละคร}

\subsubsection{การเคลื่อนไหวของผู้เล่น}

ผู้เล่นสามารถเคลื่อนที่ด้วยการเดิน กระโดด และวิ่ง โดยการวิ่งไม่สามารถวิ่งถอยหลังได้ สามารถวิ่งแล้วหันหน้ากลับมาดูข้างหลังเพียงชั่วคราวเท่านั้น

\subsubsection{การเคลื่อนไหวของสัตว์}

สัตว์เดินและวิ่งด้วยความเร็วธรรมชาติของมันขึ้นอยู่กับชนิด จะอยู่กันเป็นฝูง ตัวปีศาจมีระบบ Navigation ที่ทำให้สามารถเคลื่อนที่ไปหาหรือติดตามผู้เล่นได้

\subsection{การต่อสู้}

การต่อสู้สำหรับผู่เล่นมีไว้เพื่อป้องกันตัวเองจากการโจมตีของปีศาจเท่านั้น ไม่ได้มีไว้เพื่อฆ่าปีศาจ ผู้ที่เล่นเป็นแม่ชีเท่านั้นที่สามารถป้องกันการโจมตีของปีศาจได้ โดยหลังจากผู้เล่นป้องกันการโจมตี เขาจะบาดเจ็บ เสีย HP ผู้เล่นสามารถป้องกันการโจมตีได้เรื่อย ๆ จนกว่าจะตาย โดยปีศาจจะโจมตีทีละครั้งเท่านั้น ความถี่ในการโจมตีขึ้นอยู่กับจำนวนวงแหวนเวทย์มนต์ที่ผู้เล่นได้ทำลายไป

การโจมตีของปีศาจ ปีศาจจะเข้าหาผู้เล่นแบบลอบเร้น (Stealth) และจะโจมตีโดยการปล่อยพลังเวทย์ เสียงดนตรีและเสียงประกอบจะดังขึ้นตอนนี้ ผู้เล่นสามารถป้องกันการโจมตีได้ ถ้าหากป้องกันได้ก่อนที่ปีศาจจะโจมตี จะไม่ได้รับ Damage แต่ถ้าเริ่มโจมตีแล้วจะลด Damage เพียงกึ่งหนึ่ง

\subsection{การรักษา}

ผู้เล่นที่เล่นเป็นนายพรานสามารถรักษาผู้เล่นได้ ผู้เล่นคนนี้ต้องเก็บสมุนไพรในป่า เพื่อนำมารักษาตัวเองและผู้เล่นอีกคนหนึ่ง

\subsection{กับดัก}

กับดักในป่าสามารถทำให้แม่ชีบาดเจ็บได้เพียงแค่เดินเหยียบ ผู้ที่เล่นเป็นนายพรานสามารถมองเห็นและปลดกับดักได้ นายพรายไม่สามารถถูกกับดักทำร้ายได้

% \begin{figure}
% \begin{center}
% \includegraphics{800px-Briny_Beach.jpg}
% \end{center}
% \caption[Poem]{The Walrus and the Carpenter}
% \label{fig:walrus}
% \end{figure}
